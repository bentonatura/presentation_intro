\documentclass{standalone}
\usepackage{tikz}
%%%<
\usepackage{verbatim}
\usepackage{verbatim}
\usepackage[active,tightpage]{preview}
\usepackage{fp}


\PreviewEnvironment{tikzpicture}
\setlength\PreviewBorder{5pt}%
%%%>
\begin{comment}
:Title: A complete graph
:Tags: Foreach;Graphs;To paths
:Author: Jean-Noël Quintin
:Slug: complete-graph
\end{comment}
\usetikzlibrary[topaths]
% A counter, since TikZ is not clever enough (yet) to handle
% arbitrary angle systems.
\newcount\mycount
\begin{document}
\def\colorPallete{{"0.00 0.00 0.52", "0.00 0.77 1.00", "0.4 0.0 0.4","0.73 0.00 0.00", "0.8 0.5 0.1",  "1.00 0.98 0.00"}}
\begin{tikzpicture}[transform shape]
  %the multiplication with floats is not possible. Thus I split the loop in two.
  \foreach \number in {1,...,8}{
      % Computer angle:
        \mycount=\number
        \advance\mycount by -1
  \multiply\mycount by 45
        \advance\mycount by 0
        \pgfmathparse{\colorPallete[(\number - 1)/3]};
        \definecolor{currentColor}{rgb}{\pgfmathresult};
      \node[fill=currentColor,draw,circle,inner sep=0.2cm] (N-\number) at (\the\mycount:5.4cm) {};
    }
  \foreach \number in {9,...,16}{
      % Computer angle:
        \mycount=\number
        \advance\mycount by -1
  \multiply\mycount by 45
        \advance\mycount by 22.5
        \pgfmathparse{\colorPallete[(\number-1)/3]};
        \definecolor{currentColor}{rgb}{\pgfmathresult};
      \node[fill=currentColor,draw,circle,inner sep=0.2cm] (N-\number) at (\the\mycount:5.4cm) {};
    }
  \foreach \number in {1,...,15}{
        \mycount=\number
        \advance\mycount by 1
  \foreach \numbera in {\the\mycount,...,16}{
    \pgfmathsetmacro\num{int((\number - 1)/3)};
    \pgfmathsetmacro\numa{int(\numbera - 1)/3)};
    \pgfmathsetmacro\numsame{int(\numa - \num)}
    \ifnum\numsame=0
    \pgfmathparse{\colorPallete[\num])};
    \definecolor{currentColor}{rgb}{\pgfmathresult};
    \path (N-\number) edge[-,currentColor,line width=2pt] (N-\numbera)  ;
    \else 
    \path (N-\number) edge[-, line width=1pt, opacity=0.5] (N-\numbera)  ;
    \fi
  }
}
\end{tikzpicture}

\end{document}